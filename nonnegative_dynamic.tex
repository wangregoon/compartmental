\documentclass{paper}

\usepackage[margin=1in]{geometry} 
\usepackage{amsmath,amsthm,amssymb}
\usepackage{textcomp}
\usepackage[colorlinks,linkcolor=red,anchorcolor=green,citecolor=blue]{hyperref}
\usepackage{picture}

\newcommand{\D}{\mathcal{D}}
\newcommand{\N}{\mathbb{N}}
\newcommand{\Z}{\mathbb{Z}}
\newcommand{\R}[1]{\mathbb{R}^{#1}}
\newcommand{\RP}{\overline{\mathbb{R}}_{+}^{n}}
\newcommand{\BE}{\mathcal{B}_{\delta}(x_e)}
\newcommand{\tb}[1]{\textbf{#1}}
\newcommand{\ti}[1]{\textit{#1}}
\newcommand{\mca}[1]{\mathcal{#1}}
\newcommand{\map}[3]{{#1}:{#2}\rightarrow{#3}}
\newcommand{\rlem}[1]{Lemma.\cite{#1}}
\newcommand{\GE}{\geqslant\geqslant}
\newcommand{\LE}{\leqslant\leqslant}
%\newcommand{\rsys}[1]{Sys.\cite{#1}}

\newtheorem{problem}{Problem}
\newtheorem{thm}{Theorem}[section]
\newtheorem{prop}{Proposition}[section]
\newtheorem{lem}{Lemma}[section]
\newtheorem{cor}{Coro.}[section]
%\newtheorem{conj}[thm]{Conjecture}
\newtheorem{exer}{Exer}[section]

\theoremstyle{definition}
\newtheorem{defi}{Defination}[section]
%\newtheorem{example}{Example}[section] 

\theoremstyle{remark}
\newtheorem*{note}{Note}
\newtheorem*{remark}{\tb{Remark}}
\newtheorem*{claim}{\tb{Claim}}
\newtheorem*{examp}{\tb{Example}}
\newenvironment{solution}
               {\let\oldqedsymbol=\qedsymbol
                \renewcommand{\qedsymbol}{$\blacktriangleleft$}
                \begin{proof}[\bfseries\upshape Solution]}
               {\end{proof}
                \renewcommand{\qedsymbol}{\oldqedsymbol}}
 
\begin{document}
 
\title{Literature Review on Nonnegative Dynamics, Compartmental Modeling and Vector Dissipativity Theory}
\author{Regoon Wang, ChemE@UNSW} 


\maketitle
\begin{abstract}
Nonnegative dynamical system models are derived from mass and energy balance considerations that
involves dynamic states whose values are nonnegative. These model are widespread in biological
and ecological sciences and play a key role in the understanding of these processes.An unified
framework(linear \& nonlinear) involving compartmetal modeling, stability analysis and vector 
dissipativity theory was developed in \cite{bern_comp,hadd_non,hadd_thermo1,hadd_thermo2}. 
This work contains systematic review on these subjects and some rough ideas about their applications
in chemical process control.
\end{abstract} 

\tableofcontents
\section{Introduction}
Chemical process dynamics involves mass recycle and heat intergration. Traditionally, the model is based on state
space with interaction as the direct connection bewteen inputs and outputs of subsystem. This will lead to
a complex generalized state space model whose phyical meaning is lost. When the system grows into plantwide
complexity, it is not adequately to capture the real interconnection.

The compartmental model is a natural way to describe energy flow model involving heat flow, mass flow, work energy, 
and chemical reactions. A state-space dynamical system model that captures the key aspercts of thermodynamics, 
including its fundamental laws. So maybe, the compartmental modeling would be an alternative solution to the
plantwide complexity of chemical process. And some control stragies combining dissipativity and MPC can be
developed and form a distributed control framework.
\section{Nonnegative dynamics}
\subsection{Mathematical Preliminaries}
\begin{defi}
Let $A\in \R{m\times n}$. Then $A$ is \tb{nonnegative} (resp., \tb{positive}) if $A_{ij}\geqslant 0$ (resp.,
$A_{ij}>0$) for all $i=1,\cdots,m$ and $j=1,\cdots,n$.
\end{defi}
\begin{defi}
Let $T>0$. A real function $\map{u}{[0,T]}{\R{m}}$ is a nonegative (resp., positive) function if
$u(t)\GE 0$ (resp., $u(t)\gg 0$), which means $u_i(t)\geqslant 0$ (resp., $u_i(t)>0$) for $i=1,\cdots,m$.
\end{defi}
\begin{defi}
Let $A\in \R{n\times n}$. $A$ is a \tb{Z-matrix} if $A_{ij}\leqslant 0,i,j=1,\cdots,n,i\neq j$. A is an
\tb{M-matrix} (resp., nonsingular M-matrix) if A is a Z-matrix and all the principal minors of A are
nonnegative (resp., positive). A is \tb{essentially nonnegative} if -A is a Z-matrix, which means
$A_{ij}\geqslant 0,\forall i,j=1,\cdots,n,i\neq j$.
\end{defi}
\begin{lem} \label{lem1}
Assume A is a Z-matrix. Then the following statement are equivalent:
\begin{enumerate}
\item[(i)]    A is an M-matrix.
\item[(ii)]   $\exists \alpha>0, B\GE 0$ s.t. $\alpha > \rho(B)$ and $A=\alpha I-B$.
\item[(iii)]  Re$\lambda\geq 0$, $\lambda\in \mathrm{spec}(A)$.
\item[(iv)]   If $\lambda\in \mathrm{spec}(A)$, then either $\lambda=0$ or $\lambda>0$.
\end{enumerate}
Furthermore, in the case where A is a nonsigular Z-matrix, then the following statements are equivlent:
\begin{enumerate}
\item[(v)]    A is a nonsingular M-matrix.
\item[(vi)]   det$(A)\neq0$ and $A^{-1}\GE 0$.
\item[(vii)]  $y\in\R{n},y\GE 0$, then $\exists x\in\R{n},x\GE 0$ s.t. $Ax=y$.
\item[(viii)] $\exists x\in\R{n},x\GE 0$, s.t. $Ax\gg 0$.
\item[(ix)]   $\exists x\in\R{n},x\gg 0$, s.t. $Ax\gg 0$.
\end{enumerate}
\end{lem}
\subsection{Stability of linear case}
Consider the linear dynamical system of the form
\begin{equation}\label{sys1}
\dot{x}=Ax,\qquad x(0)=x_0, t\geqslant 0
\end{equation}
\begin{lem}\label{lem2}
Let $A\in \R{n\times n}$. $A$ is essentially nonegative iff $e^{At}$ is nonnegative for all $t\geqslant 0$.
Furthermore, if A is essentially nonnegative and $x_0\GE 0$, then $x(t)\GE 0$ and Sys.\ref{sys1} is
called \tb{linear nonnegative dynamical system}.
\end{lem}
\begin{defi}
The quilibrium solution $x(t)=x_e$ of Sys.\ref{sys1} is 
\begin{itemize}
\item \tb{Lyapunov stable} if, $\forall\epsilon>0$,$\exists \delta=\delta(\epsilon)>0$ s.t. if $x_0\in 
\BE \cap \RP$, then $x(t) \in \BE \cap \RP, t\geqslant 0$.
\item \tb{semistable} if it is Lyapunove stable and $\exists \delta>0$ s.t. if $x_0\in \BE \cap \RP$,
then $\lim\limits_{t\rightarrow\infty}x(t)$ exists and converges to a Lyapunov stable equilibrium point.
\item \tb{asymptotically stable} if it is Lyapunov stable and $\exists \delta>0$ s.t. if $x_0\in \BE \cap
\RP$, then $\lim\limits_{t\rightarrow\infty}x(t)=x_e$.
\item \tb{globally asmptotically stable} if it is asymptotically stable respect to all $x_0\in\R{n}$.
\end{itemize}
\end{defi}
\begin{thm}
Let $A\in \R{n\times n}$ be essentially nonnegative. If $\exists p,r \in\R{n}$ s.t. $p\gg 0, r\GE 0$
statisfy 
\begin{equation}\label{en_cond}
0=A^Tp+r
\end{equation}
then the following properties hold:
\begin{enumerate}
\item[(i)]    -A is an M-matrix.
\item[(ii)]   If $\lambda\in\mathrm{spec}(A)$, then either $\lambda=0$ or $\lambda>0$.
\item[(iii)]  ind$(A)\leqslant 1$, so A has generalized group inverse $A^\#$.
\item[(iv)]   A is semistable and $\lim\limits_{t\rightarrow\infty}e^{At}=I-AA^{\#}\GE 0$.
\item[(v)]    $\mathcal{R}(A)=\mathcal{N}(I-AA^{\#})$, $\mathcal{N}(A)=\mathcal{R}(I-AA^{\#})$.
\item[(vi)]   $\int_{0}^{t}e^{As}\mathrm{d}s=A^\#(e^{At}-I)+(I-AA^\#)t,t\geqslant 0$.
\item[(vii)]  $\int_{0}^{t}e^{As}\mathrm{d}sP$ exists iff $P\in\mathcal(R)(A)$.
\item[(viii)] If $P\in\mathcal(R)(A)$, then  $\int_{0}^{t}e^{As}\mathrm{d}sP=-A^\#P$.
\item[(ix)]   If $P\in\mathcal(R)(A)$ and $P\GE 0$, then $-A^\#P\GE 0$.
\item[(x)]    A is nonsigular iff -A is a nonsigular M-matrix.
\item[(xi)]   If A is nonsigular, then A is asymptotical stable and $A^{-1}\leqslant\leqslant 0$.
\end{enumerate}
\end{thm}
\begin{prop}
Suppose that $x_0\GE 0$ and $P\GE 0$, then $x_e:=\lim\limits_{t\rightarrow \infty}x(t)$ exists
iff $P\in\mathcal(R)(A)$ where $x_e=(I-AA^\#)x_0-A^\#P$. If, in addition, A is nonsingular, then
$x_e=-A^{-1}P$.
\end{prop}
\begin{thm}
Let A is essentially nonnegative. Then Sys.\ref{sys1} is asymptotically stable iff there exists a 
positive diagonal matrix P and a positive-definite matrix R s.t. 
\begin{equation}
0=A^TP+PA+R
\end{equation}
\end{thm}
\subsection{stability of nonlinear case}
Consider the nonlinear nonnegative dynamical system
\begin{equation}\label{sys:N1}
\dot{x}(t)=f(x(t)),\quad x(0)=x_0,\quad t\in[0,T_{x_0}]
\end{equation}
where $x(t)\mathcal{D}$, $\mathcal{D}$ is an open subset of $\R{n}$ containning $\RP$, 
$\map{f}{\mathcal{D}}{\R{n}}$ is locally Lipschitz. Furthermore, a subset $\D_{c}\subset \D$ is
an invariant set with respect to Sys.\ref{sys:N1} if $\D_c$ contains the orbits of all its point.
\begin{defi}[\tb{essentially nonnegative vector fields}]
let $\map{f}{\mathcal{D}}{\R{n}}$. Then $f$ is \tb{essentially nonnegative} if $f_i(x)\geqslant 0$
for all $i=1,\cdots,n$ and $x\in \RP$ s.t. $x_i=0$.
\end{defi}
\begin{prop}
$\RP$ is an invariant set with respect to Sys.\ref{sys:N1} iff $f$ is essentially nonnegative.
\end{prop}
\begin{lem}
Let $f(0)=0$ and $f$ is essentially nonnegative and continuously differentiable in $\RP$. Then,
$A:=\frac{\partial f}{\partial x}\mid_{x=0}$ is essentially nonnegative.
\end{lem}
\begin{thm}
Let $x(t)=x_e$ be an equilibrium point for Sys.\ref{sys:N1} and $f$ be essentially nonegative and 
$A=\frac{\partial f}{\partial x}\mid_{x=x_e}$. Then the following statements hold:
\begin{enumerate}
\item[(i)] If Re$\lambda < 0$, where $\lambda\in\text{spec}(A)$, then the equilibrium solution of the 
Sys.\ref{sys:N1} is asymptotically stable.
\item[(ii)] If there exits $\lambda\in\text{spec}(A)$ s.t. Re$\lambda > 0$, then the equilibrium 
solution of the Sys.\ref{sys:N1} is unstable.
\item[(iii)] Let $x_e=0$, Re$\lambda < 0$, where $\lambda\in\text{spec}(A)$, let $p\gg 0$ be s.t.
$A^Tp<<0$, and define $\D_A:=\{x\in\RP:p^Tx<\gamma\}$, where $\gamma:=\text{sup}\{\epsilon>0:
p^Tf(x)<0,\lVert x\rVert<\epsilon\}$ and $\lVert x\rVert=\sum_{i=1}^{n}p_ix_i$. Then $\D_A$ is a
subset of the domain of attraction.
\end{enumerate}
\end{thm}
\section{Compartmental modeling}
\subsection{General compartmental model}
Compartment acts like a container which allows mass or energy to flow in and out. It obeys
the universial conservation law. Compartments can be interconnected to each other (as seen in 
Fig.\ref{fig:comp}) and forms compartmetnal dynamic model which is useful in modeling 
large-scale system. As shown in Fig.\ref{fig:comp}, let $x_i(t),t\geqslant 0, i=1,\cdots,q$,
denotes the mass or energy state (and hence a nonnegative quantity) of the \ti{i}th compartment,
let $\sigma_{ii}(x_i(t))\geqslant 0$ denote the loss effect of the \ti{i}th compartment, let $w_i(t)
\geqslant 0$ denote the flus supplied to the \ti{i}th compartment, and let $\phi_{ij}(x)$ denote
the net mass flow from the \ti{j}th compartment to \ti{i}th compartment which satisfies skew-symmetry
constraint $\phi_{ij}(x)=-\phi_{ji}(x)$. Hence, the universal conservation law for the whole system
yields a nonlinear comparmental model
\begin{equation}\label{sys:n1}
\dot{x}_i(t)=-\sigma_{ii}(x_i(t))+\sum\limits_{j=1,j\neq i}^{q}\phi_{ij}(x)+w_i(t),\quad t\geqslant 0,
\quad i=1,\cdots,q
\end{equation}
Consider the linear case, let $\sigma_{ii}(x_i(t))=a_{ii}x_i(t)$,$\phi_{ij}(x)=a_{ij}x_j(t)-a_{ji}x_i(t)$,
where $a_{ii}$ denotes the loss coefficient and $a_{ij},i\neq j$ denotes the transfer coefficient, 
$a_{ij}\geqslant 0$. So the linear compartmental model can be expressed as
\begin{equation} \label{sys:L2}
\dot{x}(t)=Ax(t)+w(t),\quad x(0)=x_0, \quad t\geqslant 0
\end{equation} 
where $x(t)=[x_1(t),\cdots,x_q(t)]^T,w(t)=[w_1(t),\cdots,w_q(t)]^T$,
\begin{equation}\label{eq:A}
A_{ij}=
\begin{cases}
-\sum_{k=1}^{q}a_{ki} & \text{if } i = j \\
a_{ij} & \text{if } i \neq j
\end{cases}
\end{equation}
\begin{figure}[!h]
\centering
\setlength{\unitlength}{0.5cm}
\begin{picture}(9,8)
\put(0,3){\vector(0,-1){3}}
\put(0.2,0.5){$\sigma_{ii}(x_i(t))$}
\put(0,4){\circle{2}}
\put(-0.6,3.8){$x_i(t)$}
\put(0,8){\vector(0,-1){3}}
\put(0.2,7.5){$w_i(t)$}
\put(0.7,3.3){\vector(1,0){7.6}}
\put(3.5,2.7){$\phi_{ji}(x)$}
\put(9,3){\vector(0,-1){3}}
\put(5.9,0.5){$\sigma_{jj}(x_j(t))$}
\put(9,4){\circle{2}}
\put(8.4,3.8){$x_j(t)$}
\put(8.3,4.7){\vector(-1,0){7.6}}
\put(3.5,5.1){$\phi_{ji}(x)$}
\put(9,8){\vector(0,-1){3}}
\put(7.2,7.5){$w_j(t)$}
\end{picture}
\caption{Compartmental interconnected subsystem model}\label{fig:comp}
\end{figure}
Note that Eq.\ref{eq:A} implies that $\sum_{i=1}^{q}A_{ij}\leqslant 0$. Thus, a compartmental system (with
$w(t)\equiv 0$) satisfies $\dot{x}_i(t)\leqslant 0$ whenever $x_j(t)=0,j\neq i$, while a nonnegative system 
(with $w(t)\equiv 0$) statisfies $\dot{x}_i(t)\geqslant 0$ whenever $x_i(t)=0$. Note that A is an essentially
nonnegative matrix and the Sys.\ref{sys:L2} is a nonnegative dynamical system. Furthermore, note that 
$A^Te=[-a_{11},\cdots,-a_{qq}]^T$, and hence with $p=e=[1,\cdots,1]^T$ and $r=-A^Te\GE 0$ it follows that 
condition (\ref{en_cond}) is satisfied which implies that the Sys.\ref{sys:L2} (with $w(t)\equiv 0$) is semistable
if A is singular and asymptotically stable if A is nonsingular. In both case, $V(x)=e^Tx=\sum_{i=1}^{q}x_i$ denoting
the total mass of the system serves as a Lyapunov function for the undisturbed ($w(t)\equiv 0$)  system with
$\dot{V}=\sum_{j=1}^{q}(\sum_{i=1}^{q}A_{ij})x_j=-\sum_{i=1}^{q}a_{ii}x_i\leqslant 0, x\in \overline{\mathbb{R}}_{+}^{q}$.
Alternatively, in the case where $a_{ii}\neq 0$ and $w_i(t)\neq 0$, it follows that Sys.\ref{sys:L2} cab be equivalently
written as 
\begin{equation}
\dot{x}(t)=[J_q(x(t))-D(x(t))]\left(\frac{\partial V}{\partial x}\right)^T+w(t),\quad x(0)=x_0,\quad t\geqslant 0
\end{equation}
where $J_q(x)$ is a skew-symmetric matrix function with $J_{q(i,j)}=a_{ij}x_j-a_{ji}x_i$, and $D(x)=\text{diag}[a_{11}x_1,
\cdots,a_{qq}x_q]$ $\GE 0$. Hence, alinear compartmental system is a port-controlled hamiltonian system with a Hamiltonian
$V(x)$ representing the total mass, $D(x)$ representing the outflow dissipation, and $w(t)$ representing the supplied flux. 
%\section{Vector Lyapunove function}
\section{Vector dissipativity theory}
Dissipativity theory provides a fundamental framework for the analysis and design of control systems using an input-output
desciption based on system energy. The dissipation hypothesis on dynamical systems results in a fundamental constraint
on their dynamic behavior wherein a disspative dynamical system can only deliver a fraction of its energy to its surroundings 
and can only store a fraction of the work done to it. Since complex multi-physical system has numerous input-output 
properties related to conservation, dissipation, and trasport of mass and energy, it seems natural to extend dissipativity 
theory to nonnegative and compartmental models which themselves behave in accordance to conservation laws. Specifically, 
consider the dynamical systems of the from
\begin{align}\label{sys:d1} 
\dot{x}(t)&=f(x)+g(x)u, \quad x(0)=x_0, \quad t\geqslant 0 \\
y(t)&=h(x)+j(x)u \nonumber
\end{align}
\begin{defi}
The Sys.\ref{sys:d1} is nonnegative if for every $x_0\in\RP$ and $u(t)\GE 0$, the solution $x(t)$ and the output $y(t)$ are
nonnegative.
\end{defi}
\begin{prop}
If $f$ is essentially nonnegative, $h(x),g(x),j(x)\GE 0, x\in \RP$, then Sys.\ref{sys:d1} is nonnegative.
\end{prop}
\begin{defi}
The Sys.\ref{sys:d1} is exponentially dissipative (resp., dissipative) with respect to the supply rate $s:\overline{\mathbb{R}}_{+}^{m}
\times\overline{\mathbb{R}}_{+}^{l}\rightarrow \mathbb{R}$ if there exists a continuous nonnegative-definite function 
$V_s:\overline{\mathbb{R}}_{+}^{n}\rightarrow \overline{\mathbb{R}}_{+}$ called a storage function and a scalar $\epsilon>0$
(resp., $\epsilon=0$) s.t. $V_s(0)=0$ and the dissipation inequality
\begin{equation}
e^{\epsilon t_2}V_s(x(t_2))\leqslant e^{\epsilon t_1}V_s(x(t_1)) + \int_{t_1}^{t_2}e^{\epsilon t}s(u(t),y(t))\text{d}t, \quad t_2\geqslant t_1
\end{equation}
If $V_s(\cdot)$ is continuously differentiable, then the dissipation inequality is equivalent to 
\begin{equation}
\dot{V}_s(x(t))+\epsilon V_s(x(t))\leqslant s(u(t),y(t)), \quad t\geqslant 0
\end{equation}
\end{defi}
\begin{defi}
A nonnegative dynamical system is zero-state observable if for all $x\in\overline{\mathbb{R}}_{+}$, $u(t)\equiv 0$ and $y(t)
\equiv 0$ implies $x(t)\equiv 0$. A nonnegative dynamical system is reachable if for all $x\in\mathbb{R}_+^n$, there exist a finite time
$t_i\leqslant 0$, square integrable input $u(t)$ defined on $[t_i,0]$, s.t. the state $x(t)$ can be driven from $x(t_i)=0$ to $x(0)=x_0$.
\end{defi}
\begin{thm}
The Sys.\ref{sys:d1} is exponentially dissipative (resp., dissipative) with respect to the supply rate $s(u,y)=q^Ty+r^Tu$ iff
there exists functions $V_s,l,W:\overline{\mathbb{R}}_{+}^n\rightarrow \overline{\mathbb{R}}_{+}$, and a scalar $\epsilon>0$
(resp., $\epsilon =0$) s.t. $V_s(\cdot)$ is continuously differentiable, $V_s(0)=0$, and for all $x\in\overline{\mathbb{R}}_{+}^n$,
\begin{align}
0&=V'_s(x)f(x)+\epsilon V_s(x)-q^Th(x)+l(x) \\
0&=V'_s(x)g(x)-q^Tj(x)-r^T+W^T(x) \nonumber
\end{align}
\end{thm}
We begin by considering the nonnegative dynamical system with the nonlinear nonegative dynamical feedback system given by
\begin{align}\label{sys:dc}
\dot{x}_c(t)&=f_c(x_c(t))+g_c(x_c(t))u_c(t),\quad x_c(0)=x_{c0},\quad t\geqslant 0\\
y_c(t)&=h_c(x_c(t)) \nonumber
\end{align}
where $f_c$ is essentially nonnegative, $g_c(x_c),h_c(x_c)\GE 0$.
\begin{thm}
Consider the Sys.\ref{sys:d1} and Sys.\ref{sys:dc}, assume Sys.\ref{sys:d1} is dissipative with respect to the linear supply rate
$s(u,y)=q^Ty+r^Tu$ and with a positive-definite storage function $V_s(\cdot)$, and assume that Sys.\ref{sys:dc} is dissipative 
with respect to the linear supply rate $s_c(u_c,y_c)=q_c^Ty_c+r_c^Tu_c$ and with a positive-definite storage function $V_{sc}(\cdot)$.
Then the following statements hold:
\begin{enumerate}
\item[(i)]   If there exists a scalar $\sigma>0$ s.t. $q+\sigma q_c\LE 0$ and $r+\sigma q_c\LE 0$, then the positive feedback interconnection
is Lyapunove stable.
\item[(ii)]  If these two systems are zero-stable observable and there exists a scalar $\sigma>0$ s.t. $q+\sigma r_c<<0$ and 
$r+\sigma q_c<<0$, then the positive feedback interconnection is asympototically stable.
\item[(iii)] If Sys.\ref{sys:d1} is zero-state observable and Sys.\ref{sys:dc} is exponentially dissipative, and there exists 
a scalar $\sigma>0$ s.t. $q+\sigma q_c\LE 0$ and $r+\sigma q_c\LE 0$, then the positive feedback interconnection is asympototically stable.
\item[(iv)]  If Sys.\ref{sys:d1} is exponentially dissipative, Sys.\ref{sys:dc} is exponentially dissipative and there exists 
a scalar $\sigma>0$ s.t. $q+\sigma q_c\LE 0$ and $r+\sigma q_c\LE 0$, then the positive feedback interconnection is asympototically stable.
\end{enumerate}
\end{thm}
Consider the feedback nonnegative time-varying input nonlinearity $\sigma(\cdot,\cdot)\in\Phi$,where
\begin{align}
\Phi:=&\lbrace\sigma:\overline{\mathbb{R}}_{+}\times\overline{\mathbb{R}}_{+}^l\rightarrow \overline{\mathbb{R}}_{+}^m: 
\sigma(.,0)=0,\text{ }0\GE\sigma(t,y)\GE My,\text{ } y\in\overline{\mathbb{R}}_{+}^l, \nonumber \\
\qquad &\text{a.e. } t\geqslant 0, \text{ and } \sigma(\cdot,y) \text{ is Lebesgue measurable,}, M>>0 \rbrace
\end{align}
\begin{thm}
Consider Sys.\ref{sys:d1} is zero-state observable and exponentially dissipative with respect to the supply rate $s(u,y)=e^Tu-e^TMy$, where
$M>>0$. Then the positive feedback interconnection of Sys.\ref{sys:d1} and $\sigma(\cdot,\cdot)$ is globally asymptotically stable.
\end{thm}
\section{Application in Chemical Process Control}

\bibliographystyle{unsrt}
\bibliography{nonnegative_dynamic}

\end{document}