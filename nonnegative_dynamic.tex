\documentclass{paper}

\usepackage[margin=1in]{geometry} 
\usepackage{amsmath,amsthm,amssymb}
\usepackage{textcomp}
\usepackage[colorlinks,linkcolor=red,anchorcolor=green,citecolor=blue]{hyperref}
\usepackage{picture}

\newcommand{\D}{\mathcal{D}}
\newcommand{\N}{\mathbb{N}}
\newcommand{\Z}{\mathbb{Z}}
\newcommand{\R}[1]{\mathbb{R}^{#1}}
\newcommand{\RP}{\overline{\mathbb{R}}_{+}^{n}}
\newcommand{\BE}{\mathcal{B}_{\delta}(x_e)}
\newcommand{\tb}[1]{\textbf{#1}}
\newcommand{\ti}[1]{\textit{#1}}
\newcommand{\mca}[1]{\mathcal{#1}}
\newcommand{\map}[3]{{#1}:{#2}\rightarrow{#3}}
\newcommand{\rlem}[1]{Lemma.\cite{#1}}
\newcommand{\GE}{\geqslant\geqslant}
\newcommand{\LE}{\leqslant\leqslant}
%\newcommand{\rsys}[1]{Sys.\cite{#1}}

\newtheorem{problem}{Problem}
\newtheorem{thm}{Theorem}[section]
\newtheorem{prop}{Proposition}[section]
\newtheorem{lem}{Lemma}[section]
\newtheorem{cor}{Coro.}[section]
%\newtheorem{conj}[thm]{Conjecture}
\newtheorem{exer}{Exer}[section]

\theoremstyle{definition}
\newtheorem{defi}{Defination}[section]
%\newtheorem{example}{Example}[section] 

\theoremstyle{remark}
\newtheorem*{note}{Note}
\newtheorem*{remark}{\tb{Remark}}
\newtheorem*{claim}{\tb{Claim}}
\newtheorem*{examp}{\tb{Example}}
\newenvironment{solution}
               {\let\oldqedsymbol=\qedsymbol
                \renewcommand{\qedsymbol}{$\blacktriangleleft$}
                \begin{proof}[\bfseries\upshape Solution]}
               {\end{proof}
                \renewcommand{\qedsymbol}{\oldqedsymbol}}
 
\begin{document}
 
\title{Literature Review on Nonnegative Dynamics, Compartmental Modeling and Vector Dissipativity Theory}
\author{Regoon Wang, ChemE@UNSW} 


\maketitle
\begin{abstract}
Nonnegative dynamical system models are derived from mass and energy balance considerations that
involves dynamic states whose values are nonnegative. These model are widespread in biological
and ecological sciences and play a key role in the understanding of these processes.An unified
framework(linear \& nonlinear) involving compartmetal modeling, stability analysis and vector 
dissipativity theory was developed in \cite{bern_comp,hadd_non,hadd_thermo1,hadd_thermo2}. 
This work contains systematic review on these subjects and some rough ideas about their future
applications in chemical process control.
\end{abstract} 

\tableofcontents
\section{Introduction}
Chemical process dynamics involves mass recycle and heat intergration. Traditionally, the model is based on state
space with interaction modeled as the direct connection bewteen inputs and outputs of subsystem. This will lead to
a complex generalized state space model whose phyical meaning is lost. When the system grows into plantwide
complexity, it is not adequately to capture the real interconnection.

The compartmental model is a natural way to describe energy flow model involving heat flow, mass flow, work energy, 
and chemical reactions. A state-space dynamical system model that captures the key aspects of thermodynamics, 
including its fundamental laws. So maybe, the compartmental modeling would be an alternative solution to the
plant-wide complexity of chemical process. And the control approach combining dissipativity and MPC can be
developed and form a distributed control framework.
\section{Nonnegative Dynamics}
\subsection{Mathematical Preliminaries}
\begin{defi}
Let $A\in \R{m\times n}$. Then $A$ is \tb{nonnegative} (resp., \tb{positive}) if $A_{ij}\geqslant 0$ (resp.,
$A_{ij}>0$) for all $i=1,\cdots,m$ and $j=1,\cdots,n$.
\end{defi}
\begin{defi}
Let $T>0$. A real function $\map{u}{[0,T]}{\R{m}}$ is a nonnegative (resp., positive) function if
$u(t)\GE 0$ (resp., $u(t)\gg 0$), which means $u_i(t)\geqslant 0$ (resp., $u_i(t)>0$) for $i=1,\cdots,m$.
\end{defi}
\begin{defi}
Let $A\in \R{n\times n}$. $A$ is a \tb{Z-matrix} if $A_{ij}\leqslant 0,i,j=1,\cdots,n,i\neq j$. A is an
\tb{M-matrix} (resp., nonsingular M-matrix) if A is a Z-matrix and all the principal minors of A are
nonnegative (resp., positive). A is \tb{essentially nonnegative} if -A is a Z-matrix, which means
$A_{ij}\geqslant 0,\forall i,j=1,\cdots,n,i\neq j$.
\end{defi}
\begin{lem} \label{lem1}
Assume A is a Z-matrix. Then the following statement are equivalent:
\begin{enumerate}
\item[(i)]    A is an M-matrix.
\item[(ii)]   $\exists \alpha>0, B\GE 0$ s.t. $\alpha > \rho(B)$ and $A=\alpha I-B$.
\item[(iii)]  Re$\lambda\geq 0$, $\lambda\in \mathrm{spec}(A)$.
\item[(iv)]   If $\lambda\in \mathrm{spec}(A)$, then either $\lambda=0$ or $\lambda>0$.
\end{enumerate}
Furthermore, in the case where A is a nonsingular Z-matrix, then the following statements are equivlent:
\begin{enumerate}
\item[(v)]    A is a nonsingular M-matrix.
\item[(vi)]   det$(A)\neq0$ and $A^{-1}\GE 0$.
\item[(vii)]  $y\in\R{n},y\GE 0$, then $\exists x\in\R{n},x\GE 0$ s.t. $Ax=y$.
\item[(viii)] $\exists x\in\R{n},x\GE 0$, s.t. $Ax\gg 0$.
\item[(ix)]   $\exists x\in\R{n},x\gg 0$, s.t. $Ax\gg 0$.
\end{enumerate}
\end{lem}
\subsection{Stability of linear case}
Consider the linear dynamical system of the form
\begin{equation}\label{sys1}
\dot{x}=Ax,\qquad x(0)=x_0, t\geqslant 0
\end{equation}
\begin{lem}\label{lem2}
Let $A\in \R{n\times n}$. $A$ is essentially nonnegative iff $e^{At}$ is nonnegative for all $t\geqslant 0$.
Furthermore, if A is essentially nonnegative and $x_0\GE 0$, then $x(t)\GE 0$ and Sys.\ref{sys1} is
called \tb{linear nonnegative dynamical system}.
\end{lem}
\begin{defi}
The equilibrium solution $x(t)=x_e$ of Sys.\ref{sys1} is 
\begin{itemize}
\item \tb{Lyapunov stable} if, $\forall\epsilon>0$,$\exists \delta=\delta(\epsilon)>0$ s.t. if $x_0\in 
\BE \cap \RP$, then $x(t) \in \BE \cap \RP, t\geqslant 0$.
\item \tb{semistable} if it is Lyapunove stable and $\exists \delta>0$ s.t. if $x_0\in \BE \cap \RP$,
then $\lim\limits_{t\rightarrow\infty}x(t)$ exists and converges to a Lyapunov stable equilibrium point.
\item \tb{asymptotically stable} if it is Lyapunov stable and $\exists \delta>0$ s.t. if $x_0\in \BE \cap
\RP$, then $\lim\limits_{t\rightarrow\infty}x(t)=x_e$.
\item \tb{globally asmptotically stable} if it is asymptotically stable respect to all $x_0\in\R{n}$.
\end{itemize}
\end{defi}
\begin{thm}
Let $A\in \R{n\times n}$ be essentially nonnegative. If $\exists p,r \in\R{n}$ s.t. $p\gg 0, r\GE 0$
statisfy 
\begin{equation}\label{en_cond}
0=A^Tp+r
\end{equation}
then the following properties hold:
\begin{enumerate}
\item[(i)]    -A is an M-matrix.
\item[(ii)]   If $\lambda\in\mathrm{spec}(A)$, then either $\lambda=0$ or $\lambda>0$.
\item[(iii)]  ind$(A)\leqslant 1$, so A has generalized group inverse $A^\#$.
\item[(iv)]   A is semistable and $\lim\limits_{t\rightarrow\infty}e^{At}=I-AA^{\#}\GE 0$.
\item[(v)]    $\mathcal{R}(A)=\mathcal{N}(I-AA^{\#})$, $\mathcal{N}(A)=\mathcal{R}(I-AA^{\#})$.
\item[(vi)]   $\int_{0}^{t}e^{As}\mathrm{d}s=A^\#(e^{At}-I)+(I-AA^\#)t,t\geqslant 0$.
\item[(vii)]  $\int_{0}^{t}e^{As}\mathrm{d}sP$ exists iff $P\in\mathcal(R)(A)$.
\item[(viii)] If $P\in\mathcal(R)(A)$, then  $\int_{0}^{t}e^{As}\mathrm{d}sP=-A^\#P$.
\item[(ix)]   If $P\in\mathcal(R)(A)$ and $P\GE 0$, then $-A^\#P\GE 0$.
\item[(x)]    A is nonsingular iff -A is a nonsigular M-matrix.
\item[(xi)]   If A is nonsigular, then A is asymptotical stable and $A^{-1}\leqslant\leqslant 0$.
\end{enumerate}
\end{thm}
\begin{prop}
Suppose that $x_0\GE 0$ and $P\GE 0$, then $x_e:=\lim\limits_{t\rightarrow \infty}x(t)$ exists
iff $P\in\mathcal(R)(A)$ where $x_e=(I-AA^\#)x_0-A^\#P$. If, in addition, A is nonsingular, then
$x_e=-A^{-1}P$.
\end{prop}
\begin{thm}
Let A is essentially nonnegative. Then Sys.\ref{sys1} is asymptotically stable iff there exists a 
positive diagonal matrix P and a positive-definite matrix R s.t. 
\begin{equation}
0=A^TP+PA+R
\end{equation}
\end{thm}
\subsection{stability of nonlinear case}
Consider the nonlinear nonnegative dynamical system
\begin{equation}\label{sys:N1}
\dot{x}(t)=f(x(t)),\quad x(0)=x_0,\quad t\in[0,T_{x_0}]
\end{equation}
where $x(t)\mathcal{D}$, $\mathcal{D}$ is an open subset of $\R{n}$ containing $\RP$, 
$\map{f}{\mathcal{D}}{\R{n}}$ is locally Lipschitz. Furthermore, a subset $\D_{c}\subset \D$ is
an invariant set with respect to Sys.\ref{sys:N1} if $\D_c$ contains the orbits of all its point.
\begin{defi}[\tb{essentially nonnegative vector fields}]
let $\map{f}{\mathcal{D}}{\R{n}}$. Then $f$ is \tb{essentially nonnegative} if $f_i(x)\geqslant 0$
for all $i=1,\cdots,n$ and $x\in \RP$ s.t. $x_i=0$.
\end{defi}
\begin{prop}
$\RP$ is an invariant set with respect to Sys.\ref{sys:N1} iff $f$ is essentially nonnegative.
\end{prop}
\begin{lem}
Let $f(0)=0$ and $f$ is essentially nonnegative and continuously differentiable in $\RP$. Then,
$A:=\frac{\partial f}{\partial x}\mid_{x=0}$ is essentially nonnegative.
\end{lem}
\begin{thm}
Let $x(t)=x_e$ be an equilibrium point for Sys.\ref{sys:N1} and $f$ be essentially nonnegative and 
$A=\frac{\partial f}{\partial x}\mid_{x=x_e}$. Then the following statements hold:
\begin{enumerate}
\item[(i)] If Re$\lambda < 0$, where $\lambda\in\text{spec}(A)$, then the equilibrium solution of the 
Sys.\ref{sys:N1} is asymptotically stable.
\item[(ii)] If there exits $\lambda\in\text{spec}(A)$ s.t. Re$\lambda > 0$, then the equilibrium 
solution of the Sys.\ref{sys:N1} is unstable.
\item[(iii)] Let $x_e=0$, Re$\lambda < 0$, where $\lambda\in\text{spec}(A)$, let $p\gg 0$ be s.t.
$A^Tp<<0$, and define $\D_A:=\{x\in\RP:p^Tx<\gamma\}$, where $\gamma:=\text{sup}\{\epsilon>0:
p^Tf(x)<0,\lVert x\rVert<\epsilon\}$ and $\lVert x\rVert=\sum_{i=1}^{n}p_ix_i$. Then $\D_A$ is a
subset of the domain of attraction.
\end{enumerate}
\end{thm}
\section{Compartmental Modeling}
\subsection{General Compartmental Model}
Compartment acts like a container which allows mass or energy to flow in and out. It obeys
the universal conservation law. Compartments can be interconnected to each other (as seen in 
Fig.\ref{fig:comp}) and forms compartmetnal dynamic model which is useful in modeling 
large-scale system. As shown in Fig.\ref{fig:comp}, let $x_i(t),t\geqslant 0, i=1,\cdots,q$,
denotes the mass or energy state (and hence a nonnegative quantity) of the \ti{i}th compartment,
let $\sigma_{ii}(x_i(t))\geqslant 0$ denote the loss effect of the \ti{i}th compartment, let $w_i(t)
\geqslant 0$ denote the flus supplied to the \ti{i}th compartment, and let $\phi_{ij}(x)$ denote
the net mass flow from the \ti{j}th compartment to \ti{i}th compartment which satisfies skew-symmetry
constraint $\phi_{ij}(x)=-\phi_{ji}(x)$. Hence, the universal conservation law for the whole system
yields a nonlinear comparmental model
\begin{equation}\label{sys:n1}
\dot{x}_i(t)=-\sigma_{ii}(x_i(t))+\sum\limits_{j=1,j\neq i}^{q}\phi_{ij}(x)+w_i(t),\quad t\geqslant 0,
\quad i=1,\cdots,q
\end{equation}
Consider the linear case, let $\sigma_{ii}(x_i(t))=a_{ii}x_i(t)$,$\phi_{ij}(x)=a_{ij}x_j(t)-a_{ji}x_i(t)$,
where $a_{ii}$ denotes the loss coefficient and $a_{ij},i\neq j$ denotes the transfer coefficient, 
$a_{ij}\geqslant 0$. So the linear compartmental model can be expressed as
\begin{equation} \label{sys:L2}
\dot{x}(t)=Ax(t)+w(t),\quad x(0)=x_0, \quad t\geqslant 0
\end{equation} 
where $x(t)=[x_1(t),\cdots,x_q(t)]^T,w(t)=[w_1(t),\cdots,w_q(t)]^T$,
\begin{equation}\label{eq:A}
A_{ij}=
\begin{cases}
-\sum_{k=1}^{q}a_{ki} & \text{if } i = j \\
a_{ij} & \text{if } i \neq j
\end{cases}
\end{equation}
\begin{figure}[!h]
\centering
\setlength{\unitlength}{0.5cm}
\begin{picture}(9,8)
\put(0,3){\vector(0,-1){3}}
\put(0.2,0.5){$\sigma_{ii}(x_i(t))$}
\put(0,4){\circle{2}}
\put(-0.6,3.8){$x_i(t)$}
\put(0,8){\vector(0,-1){3}}
\put(0.2,7.5){$w_i(t)$}
\put(0.7,3.3){\vector(1,0){7.6}}
\put(3.5,2.7){$\phi_{ji}(x)$}
\put(9,3){\vector(0,-1){3}}
\put(5.9,0.5){$\sigma_{jj}(x_j(t))$}
\put(9,4){\circle{2}}
\put(8.4,3.8){$x_j(t)$}
\put(8.3,4.7){\vector(-1,0){7.6}}
\put(3.5,5.1){$\phi_{ji}(x)$}
\put(9,8){\vector(0,-1){3}}
\put(7.2,7.5){$w_j(t)$}
\end{picture}
\caption{Compartmental interconnected subsystem model}\label{fig:comp}
\end{figure}
Note that Eq.\ref{eq:A} implies that $\sum_{i=1}^{q}A_{ij}\leqslant 0$. Thus, a compartmental system (with
$w(t)\equiv 0$) satisfies $\dot{x}_i(t)\leqslant 0$ whenever $x_j(t)=0,j\neq i$, while a nonnegative system 
(with $w(t)\equiv 0$) satisfies $\dot{x}_i(t)\geqslant 0$ whenever $x_i(t)=0$. Note that A is an essentially
nonnegative matrix and the Sys.\ref{sys:L2} is a nonnegative dynamical system. Furthermore, note that 
$A^Te=[-a_{11},\cdots,-a_{qq}]^T$, and hence with $p=e=[1,\cdots,1]^T$ and $r=-A^Te\GE 0$ it follows that 
condition (\ref{en_cond}) is satisfied which implies that the Sys.\ref{sys:L2} (with $w(t)\equiv 0$) is semistable
if A is singular and asymptotically stable if A is nonsingular. In both case, $V(x)=e^Tx=\sum_{i=1}^{q}x_i$ denoting
the total mass of the system serves as a Lyapunov function for the undisturbed ($w(t)\equiv 0$)  system with
$\dot{V}=\sum_{j=1}^{q}(\sum_{i=1}^{q}A_{ij})x_j=-\sum_{i=1}^{q}a_{ii}x_i\leqslant 0, x\in \overline{\mathbb{R}}_{+}^{q}$.
Alternatively, in the case where $a_{ii}\neq 0$ and $w_i(t)\neq 0$, it follows that Sys.\ref{sys:L2} cab be equivalently
written as 
\begin{equation}
\dot{x}(t)=[J_q(x(t))-D(x(t))]\left(\frac{\partial V}{\partial x}\right)^T+w(t),\quad x(0)=x_0,\quad t\geqslant 0
\end{equation}
where $J_q(x)$ is a skew-symmetric matrix function with $J_{q(i,j)}=a_{ij}x_j-a_{ji}x_i$, and $D(x)=\text{diag}[a_{11}x_1,
\cdots,a_{qq}x_q]$ $\GE 0$. Hence, a linear compartmental system is a port-controlled hamiltonian system with a Hamiltonian
$V(x)$ representing the total mass, $D(x)$ representing the outflow dissipation, and $w(t)$ representing the supplied flux. 
\subsection{Interconnected Thermodynamic Systems}
To formulate state space thermodynamical model, consider the interconnected dynamical system shown in Fig.\ref{fig:comp}.
Let $E_i=x_i$ denote the energy, $S_i=w_i$ denote the external power supplied to or extracted from the subsystem, $\phi_{ij}$
denote the net instantaneous rate of energy flow from the $j$th subsystem to $i$th subsystem, and $\sigma_{ii}=a_{ii}$ denote 
the instantaneous rate of energy dissipation from the subsystem to environment. An energy balance for each subsystem yields
\begin{equation}\label{eq:pb}
\dot{E}(t)=w(E(t))-d(E(t))+S(t),\quad E(t_0)=E_0,\quad t\geqslant t_0
\end{equation}
where $w_i(E(t))=\sum_{j=1,j\neq i}^{q}\phi_{ij}(E)$ and $d_i(E)=\sigma_{ii}(E)$. 
Since thermodynamic compartmental model involves intercompartmental flows representing energy transfer between compartments, 
we can use graph-theoretic notions with undirected graph topologies to capture the system interconnections. We define a
connectivity matrix $C\in\R{q\times q}$ s.t. for $i\neq j$, $C_{ij}:=0$ if $\phi_{ij}(E)\equiv 0$ and $C_{ij}:=1$ otherwise,
and $C_{ii}:=-\sum_{k=1,k\neq i}^{q}C_{ki}$. Recall that if rank$C=q-1$, then system is strongly connected and energy exchange
is possible between any two subsystems.
\begin{defi}\label{def:entropy}
A continuously differentiable, strictly concave function $\SE:\overline{\mathbb{R}}_{+}^{q}\rightarrow \mathcal{R}$ is 
called the entropy function of the system if
\begin{equation}\label{eq:t2}
\left(\frac{\partial \SE}{\partial E_i}-\frac{\partial \SE}{\partial E_j}\right)\phi_{ij}(E)\geqslant 0, \quad i\neq j
\end{equation}
and $\frac{\partial \SE}{\partial E_i}=\frac{\partial \SE}{\partial E_j}$ iff $\phi_{ij}(E)=0$ with $C_{ij}=1$.
\end{defi}
\begin{prop}
Consider the isolated ($S(t)\equiv 0, d(E)\equiv 0$) interconnected dynamical system with the power balance Eq.\ref{eq:pb}.
Assume that rank$C=q-1$ and there exists an entropy function $\SE$. Then, $\sum_{j=1}^{q}\phi_{ij}(E)=0$ iff $
\frac{\partial \SE}{\partial E_i}=\frac{\partial \SE}{\partial E_j}$. Furthermore, the set of nonnegative equilibrium states of
Eq.\ref{eq:pb} is given by $\varepsilon_0:=\{E\in\overline{\mathbb{R}}_{+}^{q}: \frac{\partial \SE}{\partial E_i}=
\frac{\partial \SE}{\partial E_j}\}$.
\end{prop}
\begin{thm}
Consider the isolated ($S(t)\equiv 0, d(E)\equiv 0$) interconnected dynamical system with the power balance Eq.\ref{eq:pb}.
Assume that rank$C=q-1$ and there exists an entropy function $\SE$. Then, the isolated system is globally semistable with
Lyapunov function $V(E):=\SE(E_e)-S(E)-\lambda_e(e^TE_e-e^TE)$, where $\lambda_e:=\frac{\partial \SE}{\partial E_1}(E_e)$
and $E_e=\frac{1}{q}ee^TE(t_0)$ denotes as the equivalent state.
\end{thm}
In classical thermodynamics, the reciprocal of the system temperature is defined as $T_i:=\left(\frac{\partial \SE}
{\partial E_i}\right)^{-1}$. Eq.\ref{eq:t2} is a manifestation of the second law of thermodynamics and implies that the 
energy flows from high temperature subsystem to low temperature one.
\subsection{Gibbs and Helmoholtz Free Energy}
According to the classical thermodynamics, an additional (deformation) state representing subsystem volumes in order to
introduce the notion of work into our thermodynamically consistent state space energy flow model. The rate of work done
by the subsystem to environment is denoted by $d_{wi}$, the rate of work done by its environment is denoted by $S_{wi}$,
the volume of the subsystem is $V_i$ and the pressure $p_i(E,V)$. The net work done by each subsystem on the environment
satisfies 
\begin{equation}
p_i(E,V)\text{d}V_i=(d_{wi}(E,V)-S_{wi}(t)\text{d}t
\end{equation} 
The definition of entropy in the presence of work remains the same as in Definition \ref{def:entropy} with $\SE(E)$ 
replaced by $\SE(E,V)$. Note that 
\begin{align}
\frac{\diff \SE}{\diff t}&=\frac{\partial \SE}{\partial E_i}\frac{\diff E_i}{\diff t} +\frac{\partial \SE}{\partial V_i}
\frac{\diff V_i}{\diff t}  \\
p_i&(E,V)=\left(\frac{\partial \SE}{\partial E_i}\right)^{-1}\left(\frac{\partial\SE}{\partial V_i}\right) 
\end{align}
In the presence of work energy, the power balance Eq.\ref{eq:pb} takes the new form involving energy and deformation states
\begin{eqnarray}
\dot{E}&=w(E,V)-d_w(E,V)+S_w(t)-d(E,V)+S(t) \label{eq:e1} \\ 
\dot{V}&=D(E,V)(d_w(E,V)-S_w(t)) \label{eq:v1}
\end{eqnarray}
where $D(E,V):=\text{diag}\left[\left(\frac{\partial \SE}{\partial E_i}\right)^{-1}\left(\frac{\partial\SE}{\partial V_i}\right)\right]$
and $\left(\frac{\partial \SE}{\partial V}\right)D(E,V)=\left(\frac{\partial \SE}{\partial E}\right)$. Then the fundamental 
law of thermodynamics can be developed from Eq.(\ref{eq:e1},\ref{eq:v1})
\begin{eqnarray}
\text{First Law:}&\delta U=\diff Q - \diff W \\
\text{Second Law:}&\diff \SE\geqslant \sum\limits_{i=1}^{q}\frac{\diff Q_i}{\diff T_i} 
\end{eqnarray}
Next, we define the Gibbs free energy, the Helmholtz free energy, and the enthalpy functions for the interconnected dynamical
system. The Gibbs free energy is defined by
\begin{equation}
G(E,V):=e^TE-\sum\limits_{i=1}^{q}\left(\frac{\partial \SE}{\partial E_i}\right)^{-1}\SE_i(E_i,V_i)+
\sum\limits_{i=1}^{q}\left(\frac{\partial \SE}{\partial E_i}\right)^{-1}\left(\frac{\partial\SE}{\partial V_i}\right)V_i
\end{equation}
the Helmholtz free energy is defined by
\begin{equation}
F(E,V):=e^TE-\sum\limits_{i=1}^{q}\left(\frac{\partial \SE}{\partial E_i}\right)^{-1}\SE_i(E_i,V_i)
\end{equation}
and the enthalpy is defined by
\begin{equation}
H(E,V):=e^TE+\sum\limits_{i=1}^{q}\left(\frac{\partial \SE}{\partial E_i}\right)^{-1}\left(\frac{\partial\SE}{\partial V_i}\right)V_i
\end{equation}
Note that the above definitions are consistent with the classical thermodynamic difinitions given by $G(E,V)=U+pV-TS$, $F(E,V)=U-TS$,
and $H(E,V)=U+pV$.
\subsection{Chemical Potential}
In chemical reaction, an additional mass balance is included for addressing conservation of energy as well as conservation of mass.
This additional mass conservation equation would involve the law of mass-action enforce proportionality between a particular reaction
rate and the concentrations of the reactants. Consider chemical reaction networks described by
\begin{equation}
\sum\limits_{j=1}^{q}A_{ij}X_j\overset{k_i}{\longrightarrow} \sum\limits_{j=1}^{q}B_{ij}X_j
\end{equation}
where $A_{ij},B_{ij}$ are the stoichiometric coefficients and $k_i$ denotes the reaction rate. It can be written in a compactly in 
matrix-vector form as 
\begin{equation}\label{eq:chem}
AX\overset{k}{\longrightarrow} BX
\end{equation}
Let $n_j$ denote the mole number of the $j$th species. Invoking the law of mass-action, a reaction in which all of the stoichiometric 
coefficients of the reactants are one, the rate of reaction is proportional to the product of the concentrations of the reactants, the 
species quantities change 
\begin{equation}
\dot{n}(t)=(B-A)^TKn^A(t),\quad n(0)=n_0,\quad t\geqslant t_0
\end{equation}
where $K:=\text{diag}[k_1,\cdots,k_r]$ and $n^A:=\left[\Pi_{j=1}^qn_j^{A_{1j}},\cdots,\Pi_{j=1}^qn_j^{A_{rj}}\right]^T$. Furthermore,
let $M_j>0$ denote the molar mass of the $j$th species, let $m_j=M_jn_j$ denote the mass of the $j$th species. Then, using the transformation
$m(t)=Mn(t)$ where $M:=\text{diag}[M_1,\cdots,M_q]$, the mole balance equation can be rewritten as the mass balance
\begin{equation}
\dot{m}(t)=M(B-A)^T\tilde{K}m^A(t)
\end{equation}
where $\tilde{K}:=\text{diag}\left[\frac{k_1}{\Pi_{j=1}^qM_j^{A_{1j}}},\cdots,\frac{k_1}{\Pi_{j=1}^qM_j^{A_{rj}}}\right]$. In the absence
of nuclear reactions, the total mass of the species during each reaction is conserved
\begin{equation}
e^TM(B-A)^T=0
\end{equation}
The chemical potential has a strong connection with the second law of thermodynamics in that every process in nature evolves from 
a state of higher chemical potential towards a state of lower chemical potential. It was postulated by Gibbs that the change
in energy of a homogeneous substance is proportional to the change in mass of this substance with the coefficient of proportionality
given by the chemical potential of the substance.
\begin{align}
\dot{E}&=w(E)+P(E,m)M(B-A)^T\tilde{K}m^A(t)-d(E)+S(t) \\
\dot{m}&=M(B-A)^T\tilde{K}m^A(t)
\end{align}
where $P(E,m):=\text{diag}\left[\mu_1(E,m),\cdots,\mu_q(E,m)\right]$ and $\mu_i(\cdot,\cdot)$ is the chemical potential of a unit mass.
We assume that if $E_j=0$, then $\mu_j(E,m)=0$.
\begin{note}
For the chemical reaction network with mass balance, assume that $\mu(E,m)>>0$ for all $E\neq 0$ and 
\begin{equation}
\nu(E,m)=(A-B)M\mu(E,m)\GE 0
\end{equation}
where $\mu(E,m)$ is the vector of chemical potential of the substance and $\nu(E,m)$ is the affinity vector.
\end{note}
\section{Vector Dissipativity Theory}
Dissipativity theory provides a fundamental framework for the analysis and design of control systems using an input-output
description based on system energy. The dissipation hypothesis on dynamical systems results in a fundamental constraint
on their dynamic behavior wherein a disspative dynamical system can only deliver a fraction of its energy to its surroundings 
and can only store a fraction of the work done to it. Since complex multi-physical system has numerous input-output 
properties related to conservation, dissipation, and transport of mass and energy, it seems natural to extend dissipativity 
theory to nonnegative and compartmental models which themselves behave in accordance to conservation laws. Specifically, 
consider the dynamical systems of the form
\begin{align}\label{sys:d1} 
\dot{x}(t)&=f(x)+g(x)u, \quad x(0)=x_0, \quad t\geqslant 0 \\
y(t)&=h(x)+j(x)u \nonumber
\end{align}
\begin{defi}
The Sys.\ref{sys:d1} is nonnegative if for every $x_0\in\RP$ and $u(t)\GE 0$, the solution $x(t)$ and the output $y(t)$ are
nonnegative.
\end{defi}
\begin{prop}
If $f$ is essentially nonnegative, $h(x),g(x),j(x)\GE 0, x\in \RP$, then Sys.\ref{sys:d1} is nonnegative.
\end{prop}
\begin{defi}
The Sys.\ref{sys:d1} is exponentially dissipative (resp., dissipative) with respect to the supply rate $s:\overline{\mathbb{R}}_{+}^{m}
\times\overline{\mathbb{R}}_{+}^{l}\rightarrow \mathbb{R}$ if there exists a continuous nonnegative-definite function 
$V_s:\overline{\mathbb{R}}_{+}^{n}\rightarrow \overline{\mathbb{R}}_{+}$ called a storage function and a scalar $\epsilon>0$
(resp., $\epsilon=0$) s.t. $V_s(0)=0$ and the dissipation inequality
\begin{equation}
e^{\epsilon t_2}V_s(x(t_2))\leqslant e^{\epsilon t_1}V_s(x(t_1)) + \int_{t_1}^{t_2}e^{\epsilon t}s(u(t),y(t))\text{d}t, \quad t_2\geqslant t_1
\end{equation}
If $V_s(\cdot)$ is continuously differentiable, then the dissipation inequality is equivalent to 
\begin{equation}
\dot{V}_s(x(t))+\epsilon V_s(x(t))\leqslant s(u(t),y(t)), \quad t\geqslant 0
\end{equation}
\end{defi}
\begin{defi}
A nonnegative dynamical system is zero-state observable if for all $x\in\overline{\mathbb{R}}_{+}$, $u(t)\equiv 0$ and $y(t)
\equiv 0$ implies $x(t)\equiv 0$. A nonnegative dynamical system is reachable if for all $x\in\mathbb{R}_+^n$, there exist a finite time
$t_i\leqslant 0$, square integrable input $u(t)$ defined on $[t_i,0]$, s.t. the state $x(t)$ can be driven from $x(t_i)=0$ to $x(0)=x_0$.
\end{defi}
\begin{thm}
The Sys.\ref{sys:d1} is exponentially dissipative (resp., dissipative) with respect to the supply rate $s(u,y)=q^Ty+r^Tu$ iff
there exists functions $V_s,l,W:\overline{\mathbb{R}}_{+}^n\rightarrow \overline{\mathbb{R}}_{+}$, and a scalar $\epsilon>0$
(resp., $\epsilon =0$) s.t. $V_s(\cdot)$ is continuously differentiable, $V_s(0)=0$, and for all $x\in\overline{\mathbb{R}}_{+}^n$,
\begin{align}
0&=V'_s(x)f(x)+\epsilon V_s(x)-q^Th(x)+l(x) \\
0&=V'_s(x)g(x)-q^Tj(x)-r^T+W^T(x) \nonumber
\end{align}
\end{thm}
We begin by considering the nonnegative dynamical system with the nonlinear nonegative dynamical feedback system given by
\begin{align}\label{sys:dc}
\dot{x}_c(t)&=f_c(x_c(t))+g_c(x_c(t))u_c(t),\quad x_c(0)=x_{c0},\quad t\geqslant 0\\
y_c(t)&=h_c(x_c(t)) \nonumber
\end{align}
where $f_c$ is essentially nonnegative, $g_c(x_c),h_c(x_c)\GE 0$.
\begin{thm}
Consider the Sys.\ref{sys:d1} and Sys.\ref{sys:dc}, assume Sys.\ref{sys:d1} is dissipative with respect to the linear supply rate
$s(u,y)=q^Ty+r^Tu$ and with a positive-definite storage function $V_s(\cdot)$, and assume that Sys.\ref{sys:dc} is dissipative 
with respect to the linear supply rate $s_c(u_c,y_c)=q_c^Ty_c+r_c^Tu_c$ and with a positive-definite storage function $V_{sc}(\cdot)$.
Then the following statements hold:
\begin{enumerate}
\item[(i)]   If there exists a scalar $\sigma>0$ s.t. $q+\sigma q_c\LE 0$ and $r+\sigma q_c\LE 0$, then the positive feedback interconnection
is Lyapunove stable.
\item[(ii)]  If these two systems are zero-stable observable and there exists a scalar $\sigma>0$ s.t. $q+\sigma r_c<<0$ and 
$r+\sigma q_c<<0$, then the positive feedback interconnection is asympototically stable.
\item[(iii)] If Sys.\ref{sys:d1} is zero-state observable and Sys.\ref{sys:dc} is exponentially dissipative, and there exists 
a scalar $\sigma>0$ s.t. $q+\sigma q_c\LE 0$ and $r+\sigma q_c\LE 0$, then the positive feedback interconnection is asympototically stable.
\item[(iv)]  If Sys.\ref{sys:d1} is exponentially dissipative, Sys.\ref{sys:dc} is exponentially dissipative and there exists 
a scalar $\sigma>0$ s.t. $q+\sigma q_c\LE 0$ and $r+\sigma q_c\LE 0$, then the positive feedback interconnection is asympototically stable.
\end{enumerate}
\end{thm}
Consider the feedback nonnegative time-varying input nonlinearity $\sigma(\cdot,\cdot)\in\Phi$,where
\begin{align}
\Phi:=&\lbrace\sigma:\overline{\mathbb{R}}_{+}\times\overline{\mathbb{R}}_{+}^l\rightarrow \overline{\mathbb{R}}_{+}^m: 
\sigma(.,0)=0,\text{ }0\GE\sigma(t,y)\GE My,\text{ } y\in\overline{\mathbb{R}}_{+}^l, \nonumber \\
\qquad &\text{a.e. } t\geqslant 0, \text{ and } \sigma(\cdot,y) \text{ is Lebesgue measurable,}, M>>0 \rbrace
\end{align}
\begin{thm}
Consider Sys.\ref{sys:d1} is zero-state observable and exponentially dissipative with respect to the supply rate $s(u,y)=e^Tu-e^TMy$, where
$M>>0$. Then the positive feedback interconnection of Sys.\ref{sys:d1} and $\sigma(\cdot,\cdot)$ is globally asymptotically stable.
\end{thm}
\subsection{Generalized Vector Dissipativity}
We extend the notion of dissipative dynamical systems to develop the generalized notion of vector dissipativity for large-scale nonlinear
dynamical systems. Consider the large-scale nonlinear dynamical systems of the form
\begin{align}\label{sys:d2}
\dot{x}(t)&=F(x(t),u(t)) \\
y(t)&=H(x(t),u(t))
\end{align}
where $x\in\D\subset \R{n}$,$u\in\mathcal{U}\subset\R{m}$, $y\in\mathcal{Y}\subset\R{l}$,$F:\D\times\mathcal{U}\rightarrow\R{n}$,
$\map{H}{\D\times\mathcal{U}}{\mathcal{Y}}$, $F(0,0)=0$. Here, assume that it represents a large-scale dynamical system composed of q
interconnected subsystems
\begin{align}
F_i(x,u_i)&=f_i(x_i)+I_i(x)+G_i(x_i)u_i \\
H_i(x_i,u_i)&=h_i(x_i)+J_i(x_i)u_i
\end{align}
\begin{defi}
The Sys.\ref{sys:d2} is vector dissipative with respect to the vector supply rate $S(u,y)$ if there exists a continuous, nonnegative
definite vector function $V_s$, called a vector storage function, and an essentially nonnegative dissipation matrix W s.t $V_s(0)=0$,
W is semistable, and the vector dissipation inequality is satisfied.
\begin{equation}
V_s(x(T))\LE e^{W(T-t_0)}V_s(x(t_0))+\int_{t_0}^{T}e^{W(T-t)}S(u(t),y(t))\diff t
\end{equation}
And define the vector available storage of the large-scale system by
\begin{equation}
V_a(x_0):=\underset{T\geqslant t_0,u(\cdot)}{\text{sup}}\left[-\int_{t_0}^{T}e^{-W(t-t_0)}S(u(t),y(t))\diff t\right]
\end{equation}
where $x(t_0)=x_0$. And the concept of vector required supply is defined by
\begin{equation}
V_r(x_0):=\underset{T\geqslant -t_0,u(\cdot)}{\text{inf}}\left[\int_{-T}^{t_0}e^{-W(t-t_0)}S(u(t),y(t))\diff t\right]
\end{equation}
where $x(-T)=0,x_(t_0)=x_0$.
\end{defi}
\begin{lem}
The system is vector dissipative with respect to the vector supply rate $S(u,y)$ iff 
\begin{equation}
\dot{V}_s(x(t))\LE WV_s(x(t))+S(u(t),y(t))
\end{equation}
\end{lem}
\begin{lem}\label{lem:diss_cond}
Assume each disconnected subsystem is exponentially dissipative with respect to supply rate $s_i(u_i,y_i)$ and with a continuously
differential storage function $v_{si}$. Furthermore, assume that the interconnection functions $I_i:\D\rightarrow \R{n_i}$ art s.t.
\begin{equation}\label{eq:diss_cond}
v'_{si}(x_i)I_i(x)\leqslant \sum\limits_{j=1}^{q}\xi_{ij}(x)v_{sj}(x_j)
\end{equation}
where $\xi_{ij}$ are bounded functions. If W is semistable with
\begin{equation}
W_{ij}=
\begin{cases}
-\varepsilon_i+\alpha_{ii} & \text{if } i = j \\
\alpha_{ij} & \text{if } i \neq j
\end{cases}
\end{equation}
where $\varepsilon_i > 0$ and $\alpha_{ij}:=\text{max}\{0,\sup_{x\in\D}\xi_{ij}(x)\}$. Then the large-scale system is vector 
dissipativity.
\end{lem}
\begin{thm}
Consider the large-scale system with $S,W$, then
\begin{equation} \label{cond:ds}
\int_{t_0}^{T}e^{W(t-t_0)}S(u(t),y(t))\diff t\GE 0
\end{equation}
for $x(t_0)=0$ iff $V_a(0)=0$ and $V_a(x)$ is finite. Moreover, if (\ref{cond:ds}) holds, then $V_a(x)$ is a vector storage function.
\end{thm}
\begin{thm}
Assume the system completely reachable, then it is vector dissipative with respect to the vector supply rate $S(u,y)$ iff
\begin{equation} \label{cond:ds2}
0\LE V_r(x)\LE\infty
\end{equation} 
Moreover, if (\ref{cond:ds2}) holds, then $V_r(x)$ is a vector storage function. Finally, if $V_a(x)$ is a vector available storage 
function, then
\begin{equation}
0\LE V_a(x)\LE V_s(x)\LE V_r(x)\LE \infty
\end{equation}
And let $M=\text{diag}[\mu_1,\cdots,\mu_q], 0\leqslant \mu_i \leqslant 1$, then every vector storage function has the form
\begin{equation}
V_s(x)=MV_a(x)+(I_q-M)V_r(x)
\end{equation} 
\end{thm}
\begin{thm}
Consider the system which zero-state observable and vector dissipative. Furthermore, assume W has an non-positive eigenvalue $-\alpha$ 
and positive vector $p$. In addition, assume that there exist function $\map{k_i}{\mathcal{Y}_i}{\mathcal{U}_i}$ s.t. $k_i(0)=0$ and 
$S_i(k_i(y_i),y_i)<0$. Then for all vector storage function $V_s$ the global storage function $v_s(x):=p^TV_s(x)$ is positive definite.
\end{thm}
\subsection{Thermodynamics and Vector Dissipativity}
The fundamental and unifying concept in the analysis and control design of large-scale systems is the concept of energy. The energy
of a state of a dynamical system is the measure of its ability to produce changes in its own state as well as changes in the system 
states of its surroundings. As in thermodynamic systems, dynamical systems exhibit energy that becomes unavailable to do useful work. 
\begin{defi}
The large-scale system \ref{sys:d2} is vector dissipative with respect to the vector supply rante $S(u,y)$ if there exists a
continuous nonnegative definite vector function $V_s$ called a vector storage function, and a class $\mathcal{W}$ function $w$ s.t.
$V_s(0)=0,w(0)=0$, the zero solution $r(t)\equiv 0$ to the comparison system 
\begin{equation}
\dot{r}(t)=w(r(t))
\end{equation}
is Lyapunove stable and the vector dissipation inequality is satisfied.
\begin{equation}
\dot{V_s}(x(t))\LE w(V_s(x(t)))+S(u(t),y(t))
\end{equation}
And the dissipation inequality can be rewritten in a power balance equation.
\begin{equation}
\dot{V_s}(x(t))= w(V_s(x(t)))-d(V_s(x(t)))+S(u(t),y(t))
\end{equation}
\end{defi}
\begin{assume}\label{assume:1}
The large-scale system from viewpoint of energy should meet following 3 assumption:
\begin{enumerate}
\item[(i)] The energy flow of the interconnection is conserved.
\begin{equation}
w_i(r)=\sum\limits_{j=1,j\neq i}^{q}\phi_{ij}(r)=\sum\limits_{j=1,j\neq i}^{q}[\sigma_{ij}(r)-\sigma_{ji}(r)]
\end{equation}
\item[(ii)] For the connected system, there is no energy exchange iff the two subsystem have equal energetic.
\begin{equation}
C_{ij}=1,\quad\phi_{ij}(V_s)=0\quad\text{iff}\quad V_{si}=V_{sj}
\end{equation}
\item[(iii)] Energy flows from high energetic subsystem to less energetic subsystems.
\begin{equation}
(V_{si}-V_{sj})\phi_{ij}(V_s)\leqslant 0
\end{equation}
\end{enumerate}
\end{assume}
\begin{defi}
The entropy of large scale system can be defined as $\SE(V_s)=\tb{e}^T\ln (c\tb{e}+V_s)$, $c>0$.
\end{defi}
\begin{defi}
The ectropy of large scale system can be defined as $\varepsilon(V_s)=\frac{1}{2}V_s^TV_s$.
\end{defi}
\begin{note}
\tb{Entropy} can be thought of as a measure of the tendency of a large-scale dynamical system to lose the
ability to do useful work and lose order and to settle to a more homogeneous state. Entropy is a measure of
disorder. \tb{Ectropy} is a measure of the extent to which the system energy deviates from a a homogeneous 
state. Thus, ectropy it the dual of entropy and is measure of the tendency of a large-scale dynamical system 
to do useful work and grow more organized.
\end{note}
\begin{thm}
Consider the large-scale system with assumption.\ref{assume:1}, the entropy and ectropy satisfy:
\begin{align}
\SE(V_s(t_2))&\geqslant \SE(V_s(t_1))+\int_{t_1}^{t_2}\tb{e}^TQ(t)\diff t \\
\varepsilon(V_s(t_2))&\leqslant\varepsilon(V_s(t_1))+\int_{t_1}^{t_2}V_s^T(t)S(t)\diff t
\end{align}
where $Q:=\left[\frac{s_1-\sigma_{11}(V_s)}{c+V_{s1}},\cdots,\frac{s_q-\sigma_{qq}(V_s)}{c+V_{sq}}\right]$
\end{thm}
\begin{lem}
Let $\D_c:=\{V_s:\tb{e}^TV_s=\beta>0\}$, then 
\begin{equation}
\underset{V_s\in\D_c}{\arg\min}(\varepsilon(V_s))=\underset{V_s\in\D_c}{\arg\max}(\SE(V_s))=V_s^*=\frac{\beta}{q}\tb{e}
\end{equation}
\end{lem}
\section{Future Application in Chemical Process Control}
When applying the dissipativity theory to the nonlinear chemical process control, the most difficult part is
how to find the global dissipativity condition.  

In Denny's approach, the system dissipativity analysis in performed on linearized system to determine the candidate 
dissipative functions. And then a locally nonlinear state feedback law is designed though solving the robust nonlinear 
$\mathcal{H}_\infty$ problem, which is used for dissipativity shaping. Finally, a decentralized control frame is formed. 
It is like Down-to-Up method. 

But when applying the compartmental modeling and vector dissipavitity method, it would like an Up-to-Down approach.
The compartmental models of subsystems are obtained though first principle physical modeling. After that, the subsystem's
storage function is already obtained (the energy state of compartmental model). And by the law of physics, the disconnected
subsystems are dissipative. Then it needs to solve dissipative interconnection inequality (Lemma.\ref{lem:diss_cond}, 
Eq.\ref{eq:diss_cond}) to obtain global dissipativity condition which turns out to be a constrain on subsystem's interconnection.
In process control, the controller acts like the valve which controls the energy flow on interconnection or energy exchange
with the environment (it plays the role of energy source and sinks). So the dissipativity constraint actually is put on
the controller no matter what is the type of controller, state feedback control or MPC. This is just a rough idea. There are 
many questions about this approach such like:
\begin{enumerate}
\item[\tb{Q1.}] Is compartmental model suitable for modeling typical chemical process units such like reactor, distillation,
heat related devices, etc.
\item[\tb{Q2.}] These results are for system analysis. How to integrate them to control design?
\item[\tb{Q3.}] How to deal with some energy states which cannot be measurable, such like thermo energy?
\item[$\cdots$] \tb{Is it worth to work on it???}
\end{enumerate}    
\bibliographystyle{unsrt}
\bibliography{nonnegative_dynamic}

\end{document}