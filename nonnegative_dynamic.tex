\documentclass{paper}

\usepackage[margin=1in]{geometry} 
\usepackage{amsmath,amsthm,amssymb}
\usepackage{textcomp}
\usepackage[colorlinks,linkcolor=red,anchorcolor=green,citecolor=blue]{hyperref}
\usepackage{picture}

\newcommand{\D}{\mathcal{D}}
\newcommand{\N}{\mathbb{N}}
\newcommand{\Z}{\mathbb{Z}}
\newcommand{\R}[1]{\mathbb{R}^{#1}}
\newcommand{\RP}{\overline{\mathbb{R}}_{+}^{n}}
\newcommand{\BE}{\mathcal{B}_{\delta}(x_e)}
\newcommand{\tb}[1]{\textbf{#1}}
\newcommand{\ti}[1]{\textit{#1}}
\newcommand{\mca}[1]{\mathcal{#1}}
\newcommand{\map}[3]{{#1}:{#2}\rightarrow{#3}}
\newcommand{\rlem}[1]{Lemma.\cite{#1}}
\newcommand{\GE}{\geqslant\geqslant}
\newcommand{\LE}{\leqslant\leqslant}
%\newcommand{\rsys}[1]{Sys.\cite{#1}}

\newtheorem{problem}{Problem}
\newtheorem{thm}{Theorem}[section]
\newtheorem{prop}{Proposition}[section]
\newtheorem{lem}{Lemma}[section]
\newtheorem{cor}{Coro.}[section]
%\newtheorem{conj}[thm]{Conjecture}
\newtheorem{exer}{Exer}[section]

\theoremstyle{definition}
\newtheorem{defi}{Defination}[section]
%\newtheorem{example}{Example}[section] 

\theoremstyle{remark}
\newtheorem*{note}{Note}
\newtheorem*{remark}{\tb{Remark}}
\newtheorem*{claim}{\tb{Claim}}
\newtheorem*{examp}{\tb{Example}}
\newenvironment{solution}
               {\let\oldqedsymbol=\qedsymbol
                \renewcommand{\qedsymbol}{$\blacktriangleleft$}
                \begin{proof}[\bfseries\upshape Solution]}
               {\end{proof}
                \renewcommand{\qedsymbol}{\oldqedsymbol}}
 
\begin{document}
 
\title{Literature Review on Nonnegative Dynamics, Compartmental Modeling, Vector Lyapunov Function
and Vector Dissipativity Theory}
\author{Regoon Wang, ChemE@UNSW} 


\maketitle
\begin{abstract}
Nonnegative dynamical system models are derived from mass and energy balance considerations that
involves dynamic states whose values are nonnegative. These model are widespread in biological
and ecological sciences and play a key role in the understanding of these processes.An unified
framework(linear \& nonlinear) involving compartmetal modeling, stability analysis and vector 
dissipativity theory was developed in \cite{bern_comp,hadd_non,hadd_thermo_1,hadd_thermo_2}. 
This work contains systematic review on these subjects and some rough ideas about their applications
in chemical process control.
\end{abstract} 

\tableofcontents
\section{Introduction}
\section{Nonnegative dynamics}
\begin{defi}
Let $A\in \R{m\times n}$. Then $A$ is \tb{nonnegative} (resp., \tb{positive}) if $A_{ij}\geqslant 0$ (resp.,
$A_{ij}>0$) for all $i=1,\cdots,m$ and $j=1,\cdots,n$.
\end{defi}
\begin{defi}
Let $T>0$. A real function $\map{u}{[0,T]}{\R{m}}$ is a nonegative (resp., positive) function if
$u(t)\GE 0$ (resp., $u(t)\gg 0$), which means $u_i(t)\geqslant 0$ (resp., $u_i(t)>0$) for $i=1,\cdots,m$.
\end{defi}
\begin{defi}
Let $A\in \R{n\times n}$. $A$ is a \tb{Z-matrix} if $A_{ij}\leqslant 0,i,j=1,\cdots,n,i\neq j$. A is an
\tb{M-matrix} (resp., nonsingular M-matrix) if A is a Z-matrix and all the principal minors of A are
nonnegative (resp., positive). A is \tb{essentially nonnegative} if -A is a Z-matrix, which means
$A_{ij}\geqslant 0,\forall i,j=1,\cdots,n,i\neq j$.
\end{defi}
\begin{lem} \label{lem1}
Assume A is a Z-matrix. Then the following statement are equivalent:
\begin{enumerate}
\item[(i)]    A is an M-matrix.
\item[(ii)]   $\exists \alpha>0, B\GE 0$ s.t. $\alpha > \rho(B)$ and $A=\alpha I-B$.
\item[(iii)]  Re$\lambda\geq 0$, $\lambda\in \mathrm{spec}(A)$.
\item[(iv)]   If $\lambda\in \mathrm{spec}(A)$, then either $\lambda=0$ or $\lambda>0$.
\end{enumerate}
Furthermore, in the case where A is a nonsigular Z-matrix, then the following statements are equivlent:
\begin{enumerate}
\item[(v)]    A is a nonsingular M-matrix.
\item[(vi)]   det$(A)\neq0$ and $A^{-1}\GE 0$.
\item[(vii)]  $y\in\R{n},y\GE 0$, then $\exists x\in\R{n},x\GE 0$ s.t. $Ax=y$.
\item[(viii)] $\exists x\in\R{n},x\GE 0$, s.t. $Ax\gg 0$.
\item[(ix)]   $\exists x\in\R{n},x\gg 0$, s.t. $Ax\gg 0$.
\end{enumerate}
\end{lem}
Consider the linear dynamical system of the form
\begin{equation}\label{sys1}
\dot{x}=Ax,\qquad x(0)=x_0, t\geqslant 0
\end{equation}
\begin{lem}\label{lem2}
Let $A\in \R{n\times n}$. $A$ is essentially nonegative iff $e^{At}$ is nonnegative for all $t\geqslant 0$.
Furthermore, if A is essentially nonnegative and $x_0\GE 0$, then $x(t)\GE 0$ and Sys.\ref{sys1} is
called \tb{linear nonnegative dynamical system}.
\end{lem}
\begin{defi}
The quilibrium solution $x(t)=x_e$ of Sys.\ref{sys1} is 
\begin{itemize}
\item \tb{Lyapunov stable} if, $\forall\epsilon>0$,$\exists \delta=\delta(\epsilon)>0$ s.t. if $x_0\in 
\BE \cap \RP$, then $x(t) \in \BE \cap \RP, t\geqslant 0$.
\item \tb{semistable} if it is Lyapunove stable and $\exists \delta>0$ s.t. if $x_0\in \BE \cap \RP$,
then $\lim\limits_{t\rightarrow\infty}x(t)$ exists and converges to a Lyapunov stable equilibrium point.
\item \tb{asymptotically stable} if it is Lyapunov stable and $\exists \delta>0$ s.t. if $x_0\in \BE \cap
\RP$, then $\lim\limits_{t\rightarrow\infty}x(t)=x_e$.
\item \tb{globally asmptotically stable} if it is asymptotically stable respect to all $x_0\in\R{n}$.
\end{itemize}
\end{defi}
\begin{thm}
Let $A\in \R{n\times n}$ be essentially nonnegative. If $\exists p,r \in\R{n}$ s.t. $p\gg 0, r\GE 0$
statisfy 
\begin{equation}\label{en_cond}
0=A^Tp+r
\end{equation}
then the following properties hold:
\begin{enumerate}
\item[(i)]    -A is an M-matrix.
\item[(ii)]   if $\lambda\in\mathrm{spec}(A)$, then either $\lambda=0$ or $\lambda>0$.
\item[(iii)]  ind$(A)\leqslant 1$, so A has generalized group inverse $A^\#$.
\item[(iv)]   A is semistable and $\lim\limits_{t\rightarrow\infty}e^{At}=I-AA^{\#}\GE 0$.
\item[(v)]    $\mathcal{R}(A)=\mathcal{N}(I-AA^{\#})$, $\mathcal{N}(A)=\mathcal{R}(I-AA^{\#})$.
\item[(vi)]   $\int_{0}^{t}e^{A\\tau}\mathrm{d}\tau=A^\#(e^{At}-I)+(I-AA^\#)t,t\geqslant 0$.
\item[(vii)]  A is nonsigular iff -A is a nonsigular M-matrix.
\item[(viii)] if A is nonsigular, then A is asymptotical stable and $A^{-1}\leqslant\leqslant 0$.
\end{enumerate}
\end{thm}
\section{Compartmental modeling}
Compartment acts like a container which allows mass or energy to flow in and out. It obeys
the universial conservation law. Compartments can be interconnected to each other (as seen in 
Fig.\ref{fig:comp}) and forms compartmetnal dynamic model which is useful in modeling 
large-scale system.
\begin{figure}[h!]
\centering
\setlength{\unitlength}{0.5cm}
\begin{picture}(9,8)
\put(0,3){\vector(0,-1){3}}
\put(0.2,0.5){$a_{ii}(x_i(t))$}
\put(0,4){\circle{2}}
\put(-0.6,3.8){$x_i(t)$}
\put(0,8){\vector(0,-1){3}}
\put(0.2,7.5){$w_i(t)$}
\put(0.7,3.3){\vector(1,0){7.6}}
\put(3.5,2.7){$\phi_{ji}(x)$}
\put(9,3){\vector(0,-1){3}}
\put(5.9,0.5){$a_{jj}(x_j(t))$}
\put(9,4){\circle{2}}
\put(8.4,3.8){$x_j(t)$}
\put(8.3,4.7){\vector(-1,0){7.6}}
\put(3.5,5.1){$\phi_{ji}(x)$}
\put(9,8){\vector(0,-1){3}}
\put(7.2,7.5){$w_j(t)$}
\end{picture}
\caption{Compartmental interconnected subsystem model}\label{fig:comp}
\end{figure}
\section{Vector Lyapunove function}
\section{Vector dissipativity theory}
\section{Application in Chemical Process Control}
\bibliographystyle{unsrt}
\bibliography{nonnegative_dynamic}

\end{document}